\documentclass[10.95pt,a4paper]{letter}
\usepackage[top=.75in, bottom=.75in, left=.75in, right=0.75in]{geometry}
\usepackage{graphicx}
%\usepackage{natbib}

\address{1300 Centre Street \\ Boston, MA, 20131}
\date{March 21, 2018}
\begin{document}
%\bibliographystyle{/Users/aileneettinger/citations/Bibtex/styles/nature.bst}

\begin{letter}{}
\includegraphics[width=0.3\textwidth]{/Users/aileneettinger/Dropbox/Documents/Work/AA_heading.pdf}
\pagenumbering{gobble}

\opening{Dear Dear Editors:}
Please consider our paper, entitled ``Phenological sequences: how early-season events define those that follow," for publication as a ``Brief Communication" in the \emph{American Journal of Botany.}

Phenology, or the timing of life events such as spring flowering and leafout, has gained increasing prominence in ecology as one of the most widely documented biological impacts of anthropogenic climate change\textsuperscript{1-3}. Spring phenology has generally shifted earlier with warmer temperatures\textsuperscript{1,4}, but phenology later in the season (late spring flowering, summer fruiting, fall senescence) is less studied and may be more variable\textsuperscript{5}. Accurate predictions of phenology across the growing season are critical for forecasting important natural resources (such as nectar for pollinators or fruit for migratory birds) and for forecastiing future climate change itself, as the length of the growing season critically impacts global carbon dynamics. 

An important, but poorly studied, aspect of plant phenology is that phenological events are inherently linked through their order: leaf budburst typically occurs before flowering, and flowering always precedes fruiting. This ordering may constrain how some phenological events can respond to climate change. However, the extent to which previous phenological events are correlated with later phenological events is not known because few studies to date have integrated across multiple phenological events within individual trees during a growing season. Instead, previous studies have focused either on events related to leaf phenology (including spring budburst, leafout, and fall senescence), or reproductive events, especially flowering\textsuperscript{6}. 


In this paper, we report on observations of consecutive phenophases from the start through the end of the growing season, across 25 temperate tree species with divergent flowering phenology, grown in a common environment. We test if previous phenological events constrain later events; e.g., do late-fruiting species set fruit late in the season because they flower and leaf out late? In addition, we test whether interphase duration constrains phenology; e.g., do late-fruiting species set fruit late in the season because they require longer fruit maturation time? 

We find strong effects of both early phenology and interphase duration, highlighting the need to include previous phenological information when forecasting future phenology. Our findings have implications that are broadly important, including  for improved understanding of plant phenology and for forecasting climate change induced shifts in phenology: Our finding that early phenological events constrain later events suggests that climatic shifts in one season, even if they directly affect only one phenophase, will have cascading effects on phenology later in the season. 

We suggest as potential reviewers David Inouye, Allison Donnelly, Nicole Rafferty, Paul CaraDonna, Martin Lechowicz, and Amy Iler. Thank you for your time and consideration of our paper. 


Sincerely,\\

\includegraphics[scale=.5]{/Users/aileneettinger/Dropbox/Documents/Work/AileneEttingerSignature.png} \\
Ailene Ettinger (on behalf of both authors)
Postdoctoral Fellow, Arnold Arboretum of Harvard University \\ \& Biology Department, Tufts University


\clearpage


\noindent \emph{References}
\begin{footnotesize}
\begin{enumerate}
\item Parmesan, C. Ecological and evolutionary responses to recent climate change.  \emph{NAnnual Review of Ecology Evolution and Systematics} 37, 637-669 (2006).
\item  IPCC. Climate Change 2014: Impacts, Adaptation, and Vulnerability. Part A: Global and Sectoral Aspects. Contribution of Working Group II to the Fifth Assessment Report of the Intergovernmental Panel on Climate Change, 1132 pp (2014).
\item Root, T. L. et al. Fingerprints of global warming on wild animals and plants. Nature 421, 57-60 (2003).
\item Primack, D., Imbres, C., Primack, R., Miller-Rushing, A. & Del Tredici, P. Herbarium specimens demonstrate earlier flowering times in response to warming in Boston.  \emph{Am. J. Bot.} 91, 1260-1264 (2004).
\item Menzel, A. et al. European phenological response to climate change matches the warming pattern. Global Change Biol. 12, 1969-1976 (2006).
\item Wolkovich, E. M. & Ettinger, A. K. Back to the future for plant phenology research. \emph{New Phytol} 203, 1021-1024 (2014).
\end{enumerate}
\end{footnotesize}



\end{letter}
\end{document}
