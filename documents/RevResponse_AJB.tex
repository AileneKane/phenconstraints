\documentclass[10.95pt,a4paper]{letter}
\usepackage[top=.75in, bottom=.75in, left=.75in, right=0.75in]{geometry}
\usepackage{graphicx}
%\usepackage{natbib}

\address{1300 Centre Street \\ Boston, MA, 20131}
\date{May 31, 2018}
\begin{document}
%\bibliographystyle{/Users/aileneettinger/citations/Bibtex/styles/nature.bst}

\begin{letter}{}
\includegraphics[width=0.3\textwidth]{/Users/aileneettinger/Dropbox/Documents/Work/AA_heading.pdf}
\pagenumbering{gobble}

\opening{Dear Editors:}
Please consider our paper, entitled ``Phenological sequences: how early-season events define those that follow,"  (ELE-01149-2017) for publication as a ``Brief Communication" in the \emph{American Journal of Botany.} This manuscript has been revised to incorporate the suggestions of the referees, and this new manuscript includes new data, analyses, and text, as well as many revisions to previous figures and tables. We include a point-by-point response to reviewer comments.

Phenology, or the timing of life events such as spring flowering and leafout, has gained increasing prominence in ecology as one of the most widely documented biological impacts of anthropogenic climate change.  An important, but poorly studied, aspect of plant phenology is that phenological events are inherently linked through their order: leaf budburst typically occurs before flowering, and flowering always precedes fruiting. This ordering may constrain how some phenological events can respond to climate change. However, the extent to which previous phenological events are correlated with later phenological events is not known because few studies to date have integrated across multiple phenological events within individual trees during a growing season. In this paper, we report on observations of consecutive phenophases from the start through the end of the growing season, across 25 temperate tree species with divergent flowering phenology, grown in a common environment. We test if previous phenological events constrain later events; e.g., do late-fruiting species set fruit late in the season because they flower and leaf out late? In addition, we test whether interphase duration constrains phenology; e.g., do late-fruiting species set fruit late in the season because they require longer fruit maturation time? 

Both reviewers felt the study was interesting and makes an important contribution to the growing body of literature on plant phenology.  They provided many helpful editorial suggestions and ideas for ways to improve the clarity of figures, which we have incorporated. Reviewer 1 asked us to add a discussion of intraspecific variation in phenology (Reviewer 2 touched this as well), modify two of our figures, and add more detail to methods, all of which we have done. Reviewer 2 made a number of minor editorial suggestions, which we have followed. He also introduced several interesting nuances and thought-provoking questions, which we have tried to incorporate and address, as detailed below. 

Thank you very much for the time and consideration you have given to our paper!

Sincerely,\\

\includegraphics[scale=.5]{/Users/aileneettinger/Dropbox/Documents/Work/AileneEttingerSignature.png} \\
Ailene Ettinger (on behalf of all authors)
Postdoctoral Fellow, Arnold Arboretum of Harvard University \\ \& Biology Department, Tufts University

\end{letter}
\clearpage

\title{Response to Reviewers}
 \emph{Reviewer Comments are in italics.} Author responses are in plain text.

\section {EDITOR}
\par \emph{This is an interesting study and well-written paper.  The reviewers (particularly reviewer 1) have made some worthwhile comments that I would like to see considered by the authors.}
\par We are grateful for the kind words and constructive comments from the editor and both reviewers. We have carefully considered and worked to incorporate all comments, which we detail below.

\par \emph{I agree with the second reviewer that we need more details about how the phenophases were calculated for individual trees (line 120). }
\par We thank the editor and reviewer for pointing out the need for more detail. We have added the following to this section (Lines XX-XX): 
``On each observation day, we noted whether or not each phenophase was present. If a particular phase, such as budburst, was present, we estimated the abundance of the phenophase on the individual tree, following (Denny et al. 2014). For budburst and fruiting phenophases, abundance categories were \textless 3,11-100,101-1000,1001-10000, or \textgreater10,000 buds bursting or fruits present (Denny et al. 2014). For leafout, flowering, and leaf senescence phenophases, abundance categories were by percentage: \textless 5\% , 5-25\% ,25-49\% ,50-74\% , 75-94\% , and 95\%  or more. For leafout, we estimated the potential canopy space full with leaves (ignoring dead branches). For flowering, we estimated the percentage of flower buds that were open. For senescence, we estimated the percentage of potential canopy space full with non-green leaves (ignoring dead branches)." 

\emph{A minor point, but I noticed that the authors nearly always listed the later phenophase first in comparisons- for example, at line 171, "flowering versus budburst, flowering versus leafout, and fruiting versus flowering."  It seems more natural to me to reverse the order of these comparisons- but I leave this up to the authors.}  
\par We thank the editor for this point, and have reversed the order of these comparisons at Lines 171-172 (the phrase now reads ``budburst versus flowering, leafout versus flowering, and flowering versus fruiting''), Line 176 (``budburst versus leafout, and fruiting versus senescence''), and Line 177 (``budburst versus fruiting and leafout versus fruiting''). 
\par \emph{NOTE from the Editor: I think a version of this was submitted in LaTeX and Word, and the former (at the front of the submission) has the authors' names repeated, as reviewer Inouye noted (this may be a LaTeX glitch?).}
\par We thank the editor for pointing out this error. This was a LaTeX conversion glitch, which we have fixed. 
\\
\\
\section {REVIEWER 1}
\par \emph{This paper makes an important contribution by showing that phenological events occurring earlier in the life cycle of an organism, and their duration, affects the timing of later events.  These findings have implications for predicting and understanding phenological responses to climate change and encourage a broader consideration of factors that can explain variation in these responses, beyond the typically considered abiotic drivers. This is a well-written paper, and it was enjoyable to read (I wish I could say this more often!).}
\par We thank the reviewer for these kind words.

\par \emph{Something that is missing from this paper is a discussion of the role of intraspecific variation in sequential phenological events vs. interspecific variation.  I think the use of the across species comparison is mostly well-justified, but it would be helpful to explicitly state why intraspecific variation was not also considered.  When thinking about constraints, I would think an across-species comparison would be more noisy and that you might find even more evidence for constraints of earlier phenological events on later events when looking within species.}
\par The reviewer brings up an important and interesting point. Our primary focus in this study was interspecific variation, because interspecific variation in phenology is greater than intraspecific variation (Figure S1). We were concerned that picking up a signal within species, which had comparably small variation in phenology (e.g., budburst date within species varied by 6.8 days on average [with one third of species showing variation of one day or less], whereas budburst date across all species spanned 32 days), would require a large number of individuals per species, frequent phenology monitoring (likely daily, or more frequently, given the narrow range in intraspecific variation we observed) and greater consideration of genotypic versus environmental drivers on phenology. We agree that this would be fascinating, and is an excellent avenue for future research, which we now discuss in Lines XX-XX:

\begin{quote}
Further research could also illuminate at what other biological levels these constraints occur. We exploited the diverse species plantings of an arboreta to examine phenological constraints across species, but such constraints may also operate \emph{within} species. Phenology frequently varies across populations within a species, and local adaptation often plays a role in this variation (Rathcke and Lacey 1985). Local adaptation to diverse environments could potentially drive stronger constraints at the intraspecific level: for example, areas with short growing seasons may select for correlated stages and shorter interphase durations, while longer growing seasons may select for the opposite. Addressing this, however, requires teasing apart the influence of environment \citep[even to the microclimatic level][]{SCHWARTZ.IJBiomet.2013}, versus genotype and thus may be best accomplished in a common garden setting with multiple individuals of the same genotype. In addition, given that trees can respond  to environmental factors at even finer scales \citep[e.g., the branch level,][]{nakamura2010}, the phenological constraints we observed may affect phenology within an individual tree, as well. 
\end{quote}

\par \emph{I know that the authors used the NPN phenology classifications, but I was surprised to see that fruiting begins when the fruit is ripe (e.g., Figure 1). When I think of a plant allocating resources, however, once the flower senesces (or around this time), it allocates resources to the fruit (at least this is all still broadly allocation to reproduction).  But why not count fruiting as the days from flower end to fruit end? Do the ovaries of the flowers not begin to swell shortly after pollination? I guess this could be considered the 'fruit bud' stage, analogous to flower and leaf buds.  My concern is that this gray phase in Fig. 1 gets counted in the fruiting duration stage but then it does not get a stage of its own (i.e., no start date). If interphase duration is the number of days between the first day of flowering and the first day of ripe fruit, then fruit maturation is technically lumped in with flowering. It seems like this could affect some of the predictability of the fruiting DOY results. I'd like to see some discussion regarding this definition of the fruiting phenophase and whether it could affect the results.}  
\par We thank the reviewer for bringing up this point, and we agree that the NPN phenology classifications do not allow us to separate out exactly when fruit development begins; only when some fruits are present and when they are ripe. For some species, ovaries may swell shortly after pollination, but this may not be true for all species. Measuring this would require observation of individual flower buds, which we did not do. We now discuss this point in Lines XX-XX in the Discussion, where we write:

\par To address the reviewer's concern, we calculated interphase duration in an alternate way: as the difference between the end date of the later phase and the end date of the previous phase (i.e.`` the days from flower end to fruit end," as the reviewer suggestion). We did these calculations for n our analyses 


\par \emph{One paper that is not cited but is highly relevant is Li et al. 2016: Li, X., Jiang, L., Meng, F., Niu, H., Iler, A. M., Duan, J., et al. (2016). Responses of sequential and hierarchical phenological events to warming and cooling in alpine meadows. Nature Communications, 7, 1-8. http://doi.org/10.1038/ncomms12489}

\par We thank the reviewer for suggesting this reference, which we now cite in Line 63.

\par \emph{Line 109: Please also provide the range of the number of individuals sampled within each species. } 
\par We now say ``We selected three to five individuals of each species for the study, yielding a total of 118 individuals." (Line 109)

\par \emph{Line 127: delete 'and'}
\par We thank the reviewer for pointing out this error and have made the suggested change. 

\par \emph{Line 143: Add 'doy' in parentheses after phenological stage to remind readers what the units are.} 
\par We thank the reviewer for this suggestion, which we agree improves clarity, and have made the suggested change. 

\par \emph{Line 146-147: Some more elaboration here would be helpful.  Is this because sometimes you would get a positive value for a duration and in other species this could be a negative value (e.g., in some species, flowering starts before leaf-out and others, leaf-out starts before flowering).  Perhaps providing an example would illustrate the logic of this approach more clearly.  This randomization procedure should also make duration independent of the order of the event, which is a slightly different goal than accounting for positive vs. negative duration values (or maybe not, thus my slight confusion here).}
\par We thank the reviewer for pointing out the need for additional clarification to this section. We have added a sentence, and this section now says:
``To investigate the effect of interphase duration, separate from the constraint imposed by the inherent ordering of events, we fit models in which the interphase durations were randomized with respect to the timing of the earlier phenophase across species. Thus, this randomization procedure made the interphase duration independent of the order of the event." WE ALSO CHANGED THE FIGURE ...
\par \emph{Line 165: I find it interesting that 10 species begin to senesce before they had ripe fruits. Is this worth including in the discussion, or is this well-known for trees?} 
\par We thank the reviewer for calling attention to this phenomena and his/her interest in it. In many cases, the maturation and ripening of fruit is triggered by "termination of development of the seed structures or by metabolic activities in the fruit itself " and thus is not controlled by activities in the rest of the plant. Thus, it is common for fruit to ripen during or after leaf senescence. DO YOU ALL THINK IT  IS WORTH ADDING ANYTHING TO THE DISCUSSION?
\par \emph{Line 199: relationship is part of 'a' larger suite?}
\par We thank the reviewer for pointing out this error and have made the suggested change. 


\par \emph{Figure 1: I find this conceptual figure really helpful.  For H2, I recommend adding a label of interphase duration.  It is intuitive to think of this as the gray area, when it is really the time from start of flowering to start of fruiting.  Also add an 'f' to specify flowering instead of 'lowering' in the caption for H2.} 
\par We thank the reviewer for these suggestions, which we feel made the figure more clear. We have made the suggested changes. 


\par \emph{Figure 3:  I like this figure, but I had to spend a little time with it to really take it all in.  I wonder if it is possible to add some visual cues to help a viewer to quickly understand which hypothesis was supported.  When I was looking at it alongside reading the text, I circled the three panels that show support for the constraint hypothesis and put squares around the four panels where the regular fit is better.  I am not suggesting you do that, but just trying to clarify my suggestion.  Maybe you could put an asterisk or some other symbol in the lower right-hand corner of the panel and indicate in a legend: *constraint hypothesis (there is some white space for another legend). Or maybe you could add a light shade to the background of these panels, but that might look weird because there are not boxes around all of the panels.} 
\par We thank the reviewer for this suggestion. We added a letter to each panel in Figures 3, and have added asterisks next to the letter when Hypothesis 1 is supported. We made the same modifications to Figure 4, for Hypothesis 2. \\

\section {REVIEWER 2}

\par \emph{The manuscript takes a nice holistic perspective of phenology, using a convenient study site. In general it's well written, cites the appropriate literature, has good figures, and comes to reasonable conclusions. I've made a lot of minor editorial suggestions on the PDF, and indicated a few places where I have some questions or suggest ideas you could consider. Contact me if you need help interpreting any of my comments. 
-David Inouye}
\par We thank the reviewer for his kind words, and for the many thoughtful editorial suggestions, questions, and ideas. Below, we have copied and responded to Reviewer 2's comments made on the PDF.

\par \emph{Not clear why authors are listed twice .}
\par We thank the reviewer for pointing out this error, which we have fixed. 

\par Line 58: \emph{Preformation of buds occurs up to 4 years in advance in some species.  What effect might this have?}
\par We thank the reviewer for bringing up this fascinating question! We were unable to find preformation lengths for any of our focal species, but we agree that the preformation duration could affect future phenological responses. We now mention this in the Discussion, Line 262.

\par Line 64: \emph{Hyphenate compound adjectives}
\par We thank the reviewer for making this suggestion, and have replaced ``Early season" with ``Early-season."
 
\par Line 79: \emph{split infinitive}
\par We have replaced the phrase ``to better understand" with ``to understand."
 
\par \emph{And assuming that temperature is either consistent across years or has minimal effect.}
\par The reviewer brings up a very interesting and important point: that phenological patterns may vary from one year to the next. We discuss this in the discussion section (Lines XX-XX). 

\par Line 105: We replaced ``since" with ``because", as suggested by the reviewer.

\par Line 106: \emph{What impact might it have that some of those transplanted trees are not growing in the climate where they evolved?}
\par This is an interesting and important question. One implication is that our findings may not be predictive of patterns in the home range of our focal species. For example,  the 
day of year that we observed particular phenological events may differ from the day of year that the same event occurs in the home range, or in a different non-native population. This divergence could be due to differences in climate, photoperiod, or other environmental factors between the Arnold Arboretum and the home range. Responses could also differ between the Arboretum and the native range because of acclimatization responses or maternal effects.  Thus, although we argue in the Discussion that our findings have important implications for improved forecasting of climate change-induced shifts in phenology, we do not wish to suggest that our findings during this one year at Arnold Arboretum could be directly applied to forecasting phenology in the home ranges of our focal species. If this were a desirable application, future studies could compare responses to temperature (and other factors) of trees in their home ranges to trees at the Arnold Arboretum. SHOULD WE TOUCH ON THIS IN THE LAST PARAGRAPH OF THE DISCUSSION (before the Conclusion section)?

\par Line 118: \emph{observed with binoculars?  Looking for presence of dehiscing anthers? How tall were the trees?}

\par We did use binoculars to observe phenology, at least for some individuals, and have added this additional detail to the first sentence  of the paragraph, which now says  ``We visited each individual once every 6-10 days throughout the growing season, and, when necessary, used binoculars assist with phenology observations." Because focal individuals occurred in an arboretum, where trees are grown in the open with low branches present and other individuals are not close-by with branches obscuring views,  it was often quite easy to see buds, flowers, and leaves even without binoculars. 

\par Line 120: \emph{counting buds, flowers and leaves?}

\par We thank the reviewer for pointing out the need for additional clarity in our methodology. We have added the following to this section (Lines XX-XX): ``On each observation day, we noted whether or not each phenophase was present. If a particular phase, such as budburst, was present, we estimated the abundance of the phenophase on the individual tree, following (Denny et al. 2014). For budburst and fruiting phenophases, abundance categories were \textless 3,11-100,101-1000,1001-10000, or \textgreater10,000 buds bursting or fruits present (Denny et al. 2014). For leafout, flowering, and leaf senescence phenophases, abundance categories were by percentage: \textless 5\% , 5-25\% ,25-49\% ,50-74\% , 75-94\% , and 95\%  or more. For leafout, we estimated the potential canopy space full with leaves (ignoring dead branches). For flowering, we estimated the percentage of flower buds that were open. For senescence, we estimated the percentage of potential canopy space full with non-green leaves (ignoring dead branches)." 

\par Line 146:  \emph{Just the interphase preceding the stage of interest?}
\par We thank the reviewer for calling attention to the need for more clarity here. All previous interphase durations were used as predictors in separate models. The sentence now says:
``We therefore fit 10 different models, each with one of four phenological stages as the response variable and one of the interphase durations preceding the focal phenological stage as a predictor."
\par Line 175: \emph{It's not too surprising that these adjacent phenological stages are closely linked. E.g.,  if both are controlled by temperature, the GDD accumulation would be pretty similar for these two stages.} 
\par I WOULD LOVE SOME HELP HERE. I'M NOT QUITE SURE WHAT IS MEANT BY THIS, AND WHETHER/HOW TO RESPOND OR ADD ANYTHING.

\par Lines 192-193: \emph{on preceding stages}
\par We thank the reviewer and have made the suggested change
 
\par Lines 209-212: \emph{hyphen; that's a compound adjective}
\par We thank the reviewer for this suggestion, and have replaced ``later flowering" and ``late fruiting" with `later-flowering" and ``late-fruiting." 

\par Lines 221-222: \emph{I wonder whether differences in frost sensitivity of leaves vs. buds could play a role. }

\par Line 241: \emph{split infinitive}
\par We have removed the split infinitive, and the phrase now says: ``and highlights the need to understand how such constraints vary across years..."

\par Line 249-250: \emph{Or enough years of data to encompass significant natural variation.  That's maybe easier than manipulating tree temperatures.}
\par We thank the reviewer for this important point, and have added the following sentence:
\par Line 251: we have replaced ``climate change induced" with ``climate change-induced" as suggested by the reviewer.
\\
\par Line 262: \emph{And if any of them preform buds more than a year in advance.}
\par We thank the reviewer for making this important point. We have added a phrase to address this, so that the section now reads:
`` The ecological memory of phenology has not been quantified, but may be critical for accurate forecasting, particularly for species like \emph{Quercus rubra}, which require more than one year for fruit maturation, as well species that preform buds more than one year in advance of budburst (Diggle 1997, Klimesova and Klimes 2007)." IF POSSIBLE, FIND OUT WHICH OF OUR SPECIES PREFORM BUDS MORE THAN A YEAR IN ADVANCE, AND ADD AN EXAMPLE SPECIES HERE. (SO FAR I HAVE NOT FOUND ANY STUDIES ABOUT PREFORMATION FOR OUR FOCAL SPECIES, AND STUDIES MEMBERS OF THE SAME GENUS, E.G. Fraxinus pensylvanica, do not preform more than a year in advance)
\\
\par References: We thank the reviewer for pointing out several errors in the bibliography, where capitalization was missing. We have made the suggested changes!
\\
\par Figure 1: \emph{To me it looks like what's happening in species 2 is that flowering is spread out more than in some others, and a consequence is that some fruits (from early flowers) are developing after a relatively short interphase.  What if interphase were measured for individual flowers?  Maybe it would be much more consistent (and longer).}
\par The reviewer brings up an interesting point, and we agree that it would be valuable and fascinating to understand drivers and effects of phenology at the individual flower level. We focused on the species-level, because there is large interspecific variation in phenological patterns. Does intraspecific variation (e.g. within a population, or even within an individual, as this reviewer suggests) follow similar patterns as we have observed here? This would be an excellent avenue for future research, in our opinion, and we now mention in on Lines XX-XX.

\end{document}
